\chapter{Background}\label{chap:background}
The basis of CurbNet is semantic segmentation using convolutional neural networks.
As such, it is imperative that the reader has an understanding of the principles behind semantic segmentation and convolutional neural networks.
This chapter on the background of this work will discuss semantic segmentation, machine learning with artificial neural networks and convolutional neural networks, curb segmentation, loss functions, and network optimizers.

\section{Semantic Segmentation}\label{section:background-segmentation}
Unlike image classification, which classifies the contents of an image, semantic segmentation is the use of some algorithm to process an image and assign class labels to each individual pixel~\cite{segmentation-medium}.
This adds the ability to locate and classify multiple objects in a given scene.
For example, the ground truth segmentation of the street-level image in \figref{fig:background-raw} can be seen in \figref{fig:background-segmented}.
Each pixel of the image has been assigned a class, which is represented by the different colors in the segmented image.
This allows a computer or program to understand what objects are in the image it is shown.

\begin{figure}
    \centering
    \begin{minipage}{0.45\textwidth}
        \centering
        \includegraphics[width=0.9\textwidth]{figures/background/raw.jpg} % first figure itself
        \caption{An example street-level image as taken from a regular color camera.} \label{fig:background-raw}
    \end{minipage}\hfill
    \begin{minipage}{0.45\textwidth}
        \centering
        \includegraphics[width=0.9\textwidth]{figures/background/segmented.png} % second figure itself
        \caption{The segmented version of the image, each color representing a different class.} \label{fig:background-segmented}
    \end{minipage}
\end{figure}

By segmenting an image in this way, the program can interpret the or its environment semantically.
For example, by receiving the segmented image, a program can identify that there are line markings on the road and that there is a vehicle in front of it.
The segmentation of images in this way is essential in many robotic applications as it allows further higher level processing of the scene.

Towards the goal of this work, the segmentation of curbs and curb cuts in a scene would allow Obelix to more accurately find a path allowing for the safe traversal from sidewalk to street level via curb cuts.

In recent years, state-of-the-art methods in image segmentation have relied entirely on deep learning using CNNs to achieve better results.

\section{Artificial Neural Networks}
Artificial neural networks are a computing system inspired by biological neurons. Neural networks are comprised of neurons which are capable of taking any number of numerical inputs and outputs a numerical value. Mathematically, a single neuron is a time dependent function.
The function of a neuron $j$ receiving input $p_j(t)$ and producing output $o_j(t)$ is composed of the activation $a_j(t)$, potentially a threshold $\Theta_j$, an activation function $f$ that returns the activation at time $t + 1$, and an output function $f_{out}$.
The activation $a_j(t)$ can also be considered the neuron's state and is dependent on the time parameter $t$.
The threshold $\Theta_j$, if it exists, is fixed unless a learning function changes it.
The activation function $f$ calculates $a_j(t + 1)$ given $a_j(t)$, $\Theta_j$, and the network input $p_j(t)$ and can be defined as:
\begin{align}
	a_j(t + 1) &= f\left(a_j(t), \Theta_j, p_j(t)\right)
\end{align}
The output function $f_out$ computes $o_j(t)$ based on $a_j(t)$ and is defined as:
\begin{align}
	o_j(t) &= f_out\left(a_j(t)\right)
\end{align}
The output function is usually simply the identify function $f(x) = x$.
Many activation functions exists and are used including the identity function and the rectified linear unit (ReLU), defined as:
\begin{align}
	f(x) &= 
	\begin{cases}
		0	& \text{for } x \leq 0\\
		x	& \text{for } x > 0
	\end{cases}
\end{align}

Between each neuron in the network are connections which transfer the output of neuron $i$ to neuron $j$. Each of these connections are assigned a weight and potentially a bias term and is computed by the propagation function to provide a neuron its input $p_j(t)$. This typically is defined as:
\begin{align}
	p_j(t) &= \sum_{i}o_i(t)w_{ij} + w_{0j}~,~\text{where } w_{0j} \text{ is the bias, if it exists.}
\end{align}

Learning occurs by using an algorithm to modify the parameters of the neural network.

Deep neural networks are so called due to having multiple "hidden" layers between the input and output.
These hidden layers are not visible to the end user and their values are typically never accessed directly.

\subsection{Convolutional Neural Networks}\label{section:background-cnn}
Convolutional Neural Networks (CNNs) are a specific class of deep neural networks and have been found to perform exceptionally for image analysis, such as image classification and segmentation.
The inspiration for CNNs come from biological processes to simulate the orgainization of the visual cortex in animals~\cite{cnnbiology}.
Convolutional layers are the core build blocks of CNNs.
These convolutional layers consist of a set of learnable filters which are convolved across the entire input and computing the dot products between the different entries of the filter.
This allows the layer to activate when it detects a specific type of feature at some point in the input.\todo{convolution image}
By using multiple convolutional layers, higher level features can be extracted and a feature map generated.

For semantic scene segmentation applications, the convolutional layers that make up the CNN are called the feature encoder.
The output of these convolutional layers are then fed into one or more fully connected linear layers to produce a classification of each pixel in a secene.
These fully connected layers are called the decoder.
The resulting classification produced is a probability of each pixel being a certain class.


\section{Curbs and Curb Cuts}\label{section:background-curbs}
Curbs are the edges of roads, or the separator between the road and some other area, usually a pedestrian sidewalk, and are usually stone or concrete.
Curb cuts are "cuts" in the curb that allow a pedestrian sidewalk to have a gentle slope down to street level.
Originally, these cuts were made to allow accessibility access, especially for those requiring wheelchairs, as to a wheelchair user, curbs represent a significant barrier in terms of traversability, as was discussed in the article "Curb Cuts" by Cynthia Gorney and Delaney Hall. These curb cuts first started appearing fifty years ago from the efforts of activist Ed Roberts and his push to make city streets more accessible.
\todo{Include picture of a curb cut}

\section{Loss Functions}\label{section:background-loss}
The loss function $l$ maps the output of the neural network $\hat{y}$ and the target $y$ onto a real number and represents the "cost" of an output.
The goal of learning is to minimize this cost value, known as the network loss.
Effectively, the loss function measures the difference between the predicted value and the target.
In our case, the target value is the ground truth labeling of an image and the output is the predicted labeling from the network.
There are many different loss functions that are commonly used. For the purposes of image segmentation, the most commonly used function is cross entropy loss.

The weighted variant, known simply as weighted cross entropy loss, is the basis of the custom loss function we use for CurbNet and is defined as:
\begin{align}
	\text{loss}(x, class) &= weight[class]\left(-\log\left(\frac{\text{exp}(x[class])}{\sum_{j}\text{exp}(x[j])}\right)\right)
\end{align}
where $x$ is the predicted labeling, $class$ is the ground truth labeling, $weight$ is the weight given to each class, and $j$ is the individual pixels in $x$.

The use of cross entropy loss is to ensure that the cost increases exponentially as the predicted probability approaches zero.
This due to its use of a negative log function, whose value exponentially increases as it approaches zero.
Using the weighted version allows us to change how much each class contributes to the loss.
We do this to counteract the class imbalance caused by how small spatially the curbs are relative to the rest of the scene.

\section{Optimizers}\label{section:background-optimizers}
During training, network parameters must be updated to minimize the loss function at each iteration.
This is done by the optimizer.
The optimizer updates the network parameters in such a way as to minimize the loss function~\cite{optimizer-intro}.

Gradient Descent is one of the oldest loss functions used.
It works by calculating what a small change in each network parameter would do to the loss function, i.e. determines the optimal gradient for the current iteration.
It then adjusts the network parameters according to this gradient.
This process is repeated at each iteration to further minimize the loss function.
This process can be time consuming, especially with larger datasets.
Stochastic Gradient Descent (SGD) is a more commonly used variant of gradient descent and works similarly, but working only on randomly selected batches of the training data at each iteration~\cite{optimizer-intro}.

Adaptive Moment Estimation (Adam) is another loss function that is commonly used and the one implemented in CurbNet.
It combines concepts from Adaptive Gradient Algorithm (AdaGrad), which has per-parameter learning rates, and Root Mean Square Propagation (RMSProp), whose learning rates are adapted based on the average of the magnitude of recent gradients~\cite{adam}.
Adam utilizes the concept of momentum, which incorporates previous gradients into the current one.
It does this by calculating an exponential moving average and the square of previous gradients and using these to determine the next gradient.
The original paper that proposed Adam, titled "Adam: A method for stochastic optimization" by Diederik Kingma and Jimmy Ba, also shows that Adam allows a network to converge faster than competing optimizers, including SGD~\cite{adam}.

\section{backpropagation}\label{section:background-backpropagation}
backpropagation is a learning procedure used in artificial neural networks to adjust network weights and produce internal representations "important features in the task domain", first described the David E. Rumelhart, Geoffrey E. Hinton, and Ronald J. Williams in their paper "Learning representations by back-propagating errors"~\cite{backprop}.
backpropagation efficiently and repeatedly adjusts the network weights to minimize the network error, which is calculated by the loss function.

Calculating the network weights is done layer by layer, starting with the final output layer.
The effect the output unit has on the loss is calculated first as
\begin{align}\label{eq:loss-1}
	\frac{\partial \text{loss}}{\partial y_j}
\end{align}

where $y_j$ is the output of a single unit in the final layer.
The contribution of the input of the unit $j$ on the error can then be calculated using the value previously calculated in \eqref{eq:loss-1}
\begin{align}\label{eq:loss-2}
	\frac{\partial \text{loss}}{\partial x_j} &= \frac{\partial \text{loss}}{\partial y_j} \cdot \frac{\partial y_j}{\partial x_j}
\end{align}
We now have a description of how changing the input of unit $j$ will affect the loss.
As such, we can then also represent how changing the weight $w_{ji}$ of a connection between the unit $j$ and a unit in the previous layer $i$ will effect the error as
\begin{align}\label{eq:loss-3}
	\frac{\partial \text{loss}}{\partial w_{ji}} &= \frac{\partial \text{loss}}{\partial x_j} \cdot \frac{\partial x_j}{\partial w_{ji}}
\end{align}
In this case, the previous layer means the layer which during forwards-propagation would be calculated immediately before the layer in which unit $j$ is.
Again, due to the use of the chain rule, the previous value calculated in \eqref{eq:loss-2} can be used here.
This gives us the contribution of the weight $w_{ji}$ on the error and, by multiplying by the learning rate, the amount by which the gradient must be changed for the next training iteration.

The error contribution of the output of unit $i$ via a single successor unit $j$ can also similarly be calculated using the value from \eqref{eq:loss-2}.
\begin{align}\label{eq:loss-4}
	\frac{\partial \text{loss}}{\partial y_i} &= \frac{\partial \text{loss}}{\partial x_j} \cdot w_{ji}
\end{align}
Since unit $i$ is connected to multiple successor units, its value is a summation of its output contribution via all units $j \in J$, where $J$ is the set of all units in successive layers that $i$ is connected to.
\begin{align}
	\frac{\partial \text{loss}}{\partial y_i} &= \sum_{j \in J} \frac{\partial \text{loss}}{\partial x_j} \cdot w_{ji}
\end{align}
This calculation can be done in parallel, thanks to parallel computing architecture, completing to gradient calculations for the penultimate layer.
This procedure is then repeated for all layers, each time using the previously computed values to allow for efficient computation.

The main drawback of backpropagation is that it may approach local minima which are not the global minimum, at which point the network can potentially get stuck and become unable to improve.
The addition of more units, thus increasing the weight dimensionality, can increase the number of paths out of local minima.
Futhermore, the authors of the previously mentioned paper have also concluded that this is not a plausible model for how learning works in biological brains and, thus, that looking into other biologically based procedures may be useful.

\section{Hyperparameter Tuning}\label{section:background-hyperparameter}
Hyperparameter tuning is the process of optimizing a model to find the hyperparameters that will perform the best according to a specified metric. This metric is usually either the validation loss or the validation accuracy.
Hyperparameters are generally chosen by the researcher or using some other automated process, but is not learned as a part of the model itself as opposed to the model weights.

Methods to find the optimal hyperparameters include manual tuning, grid search, random search, genetic algorithms, bayesian optimization, and hyperband, among others.
Random and grid searches are very time-consuming and computationally expensive on all but the smallest of networks, as they potentially run less than optimal parameters and disregard previous runs.
Bayesian optimization attempts to curtail this problem by taking previous runs into account, greatly speeding up the learning process and not experimenting with less than optimal hyperparameters~\cite{bayesian}.
Hyperband is a bandit-based approach to hyperparameter optimizations that can abandon or reduce the budget of an experimental run if it deems the results to be significantly poor~\cite{hyperband}.

\subsection{Bayesian Optimization}\label{subsection:background-bayesian}
The idea behind Bayesian optimization is that the hyperparameters of a network follow a probabilistic distribution $\text{P}(y | x)$ where $y$ is the score determined by some objective function and $x$ is a set of hyperparameters.
Based on this probabilistic model, an optimal set of hyperparameters can then be calculated.
Applying these hyperparameters to the actual model the objective function can then be used to calculate the actual scores.
The probabilistic distribution is the fine tuned and the process is repeated until the maximum number of iterations is reached.

Bayesian optimization relies on the following~\cite{bayesiantds}:
\begin{enumerate}
	\itemsep-1em
	\item A clearly defined hyperparameter domain of possible configurations $\chi$ over which to search,
	\item An objective function which calculates a score using a set of hyperparameters that we wish to minimize,
	\item A surrogate model that represents the probabilistic distribution of the hyperparameters,
	\item A selection function to evalute which hyperparameters should be evaluated next, and
	\item A history of score-hyperparameter pairs used to update the surrogate model .
\end{enumerate}
The domain must be chosen by the researcher and is usually defined with regards to previous knowledge or based on related works.
The objective function is calculated by running the actual model and calculating its error using a predetermined loss function.

The surrogate model used is essentially a mapping of hyperparameters to scores from the objective function.
An example for the surrogate model is the Tree-structured Parzen Estimator (TPE)~\cite{tpe}.
TPE takes advantage of Bayes' rule and represents the probability function $\text{P}(y~|~x)$ as
\begin{align}\label{eq:bohb-1}
	\text{P}(y~|~x) &= \frac{\text{P}(x~|~y) \cdot \text{P}(x)}{\text{P}(y)}\\
	\text{with } \text{P}(x~|~y) &=
	\begin{cases}
	l(x) &\text{when } y < y^* \\
	g(x) &\text{when } y \geq y^*
	\end{cases}
\end{align}
where $y^*$ is a threshold value. $l(x)$ is thus a probability density function formed from the set of observations $x^i \in \chi$ which have been performed such that the loss of the network running with hyperparameters $x^i$ is less than $y^*$. 
Conversely, $g(x)$ is the density function formed from the remaining observations.
$y^*$ is chosen by the algorithm such that $\text{P}(y<y^*) = \gamma$ with $\gamma$ being a value chosen by the researcher.

The selection function chooses which hyperparameters $x \in \chi$ should be chosen for each successive experiment and is commonly based on expected improvement~\cite{bayesiantds}
\begin{align}
	EI_{y^*}(x) &= \int_{-\infty}^{y^*} (y^* - y) p(y~|~x)dy
\end{align}
Here, $y^*$ is again a threshold value, $x$ is the proposed set of hyperparameters, $y$ is the actual value of the objective function using hyperparameters $x$ and $p(y~|~x)$ is the surrogate probability model expressing the probability of $y$ given $x$.
Substituting the values of $p(y~|~x)$ from the surrogate function we get
\begin{align}
	EI_{y^*}(x) &= \frac{\gamma y^* l(x) - l(x) \int_{-\infty}^{y^*}p(y)dy}{\gamma l(x) + (1 - \gamma)g(x)}\\
	&\propto \left(\gamma + \frac{g(x)}{l(x)} (1 - \gamma)\right)^{-1}
\end{align}
We can see from this function that the expected improvement is proportional to $\frac{l(x)}{g(x)}$ and thus, to maximize the expected improvement, samples should be drawn from the maximum value in $l(x)$.
The selected hyperparameters are then evaluated with regards to the objective function and the results used to update the surrogate model.

\subsection{Hyperband}\label{section:background-hyperband}
Hyperband is a learning strategy that "relies on a principled early-stopping strategy to allocate resources"~\cite{hyperband}. 
It is a variation of Successive Halving, and in fact uses successive halving in its inner loop.

Successive halving randomly samples $n$ hyperparameter sets in the search domain $\chi$.
It then evaluates all of these sets for $B$ iterations to calculate the validation loss.
The lowest performing half is then discarded and the remaining evaluated for a further $B$ iterations.
This is repeated until only 1 hyperparameter set remains.

Successive halving suffers from the $n \text{ vs } \frac{B}{n}$ problem, which is whether to train more configurations $n$ or to explore fewer, but with more resources $B$.
If a larger $n$ is chosen, configurations which are slower to converge might be killed off too quickly.
If a larger $\frac{B}{n}$ is chosen, then lower performing configurations may be given more resources, thus wasting resources that could otherwise be spent on higher performing configurations~\cite{hyperband}.

Hyperband attempts to solve this issue by considering several possible values of $n$ for a fixed $B$.
The pseudocode in Algorithm \ref{algorithm:hyperband} shows how Hyperband uses an outer loop, performing essentially a grid search with multiple values of $n$ and an inner loop utilizing a method similar to successive halving but with the top $\left\lfloor n_i/\eta\right\rfloor$ instead of the top half.

\begin{algorithm}
	\caption{\textsc{Hyperband} algorithm for hyperparameter optimization}\label{algorithm:hyperband}

	\SetKwInOut{Input}{Input}
	\SetKwInOut{Output}{Output}
	\SetKw{Initialize}{Initialize}
	
	\Input{$R,~\eta~ (\text{default} \eta = 3$)}
	\Output{Configuration with the smallest intermediate loss seen so far.}
	\Initialize{$s_{max} = \left\lfloor \log_\eta(R)\right\rfloor$, $B=(s_{max} + 1)R$}
	
	\For{$s \in s_{max}, s_{max} - 1,...,0$}{
		$n = \left\lceil \frac{B}{R} \frac{\eta^s}{(s+1)}\right\rceil$\\
		$r = R\eta^{-s}$\\
		$T = $ get\_hyperparameter\_configuration$(n)$\\
		\For{$i \in \{0,...,s\}$}{
			$n_i = \left\lfloor n \eta^{-i}\right\rfloor$ \\
			$r_i = r \eta^i$ \\
			$L = $\{run\_then\_return\_validation\_loss$(t,r_i) : t \in T$\}\\
			$T = $top\_k$(T,L, \left\lfloor n_i / \eta \right \rfloor)$\\
		}
	}
\end{algorithm}

\subsection{Bayesian Optimization with Hyperband}\label{section:background-bohb}
"BOHB: Robust and Efficient Hyperparameter Optimization at Scale" by Stefan Falkner, Aaron Klein, and Frank Hutter proposes a hyperparameter optimization approach which seeks to combine the benefits of both Bayesian optimization and Hyperband~\cite{bohb}.
This approach seeks to have strong anytime performance, strong final performance, effectively use parallel resources, be easily scalable, and be robust and flexible.

BOHB works by using Hyperband to determine the number of configurations to evaluate with a given budget, but replaces the random selection of configurations with a Bayesian model based search.
The process of using Bayesian optimization for configuration sampling can be seen in Algorithm \ref{algorithm:bohb}. In this algorithm, $l'(x)$ is the $l(x)$ component of the newly updated probability density function.

Even though Hyperband also provides strong anytime performance compared to random search and Bayesian optimization, Fakner et al. observed an improvement of over 55x against a random search at larger budgets, converging to the global optimum much faster than either Hyperband or Bayesian optimization alone~\cite{bohb}.

\begin{algorithm}
	\caption{\textsc{BOHB} algorithm for hyperparameter optimization}\label{algorithm:bohb}

	\SetKwInOut{Input}{Input}
	\SetKwInOut{Output}{Output}
	\SetKw{Initialize}{Initialize}
	
	\Input{observations $D$, fraction of random runs $\rho$, percentile $q$, number of samples $N_s$, minimum number of points $N_min$ to build a model, and bandwidth factor $b_w$}
	\Output{next configuration to evaluate}
	
	\If{rand() $< \rho$}{
		\Return{random configuration}
	}
	$b = \text{arg max }\{D_b:|D_b|\geq N_min + 2\}$\\
	\If{$b=\emptyset$}{
		\Return{random configuration	
	}}
	fit P$(x~|~y)$ according to \eqref{eq:bohb-1}\\
	draw $N_s$ samples according to $l'(x)$\\	
	\Return{sample with highest ratio $l(x) / g(x)$}
\end{algorithm}

\section{Binary Dilation}\label{section:background-dilation}
Binary dilation is a basic morphological operation and is usually represented by the operator $\oplus$.
With regards to this work, binary dilation is used in our loss function.
For a given binary image viewed as an integer grid $\mathbb{Z}^d$ for some dimension $d$, let $E$ be an integer grid, $A \in E$ a binary image, and $B \in \{0,1\}^d$ a structuring element.
The binary dilation of $A$ by $B$ is then defined as:
\begin{align}
	A \oplus B &= \bigcup_{b \in B}A_b
\end{align}
where $A_b$ is the translation of $A$ by $b$.
This can be seen as extending the area of the binary image $A$ by locus of the points covered by $B$ given that $B$ has a center on the origin~\cite{morphology}.