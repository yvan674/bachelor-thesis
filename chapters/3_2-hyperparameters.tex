\section{Hyperparameter Tuning}\label{section:background-hyperparameter}
Hyperparameter tuning is the process of optimizing a model to find the hyperparameters that will perform the best according to a specified metric. This metric is usually either the validation loss or the validation accuracy.
Hyperparameters are generally chosen by the researcher or using some other automated process, but is not learned as a part of the model itself as opposed to the model weights.

Methods to find the optimal hyperparameters include manual tuning, grid search, random search, genetic algorithms, bayesian optimization, and hyperband, among others.
Random and grid searches are very time-consuming and computationally expensive on all but the smallest of networks, as they potentially run less than optimal parameters and disregard previous runs.
Bayesian optimization attempts to curtail this problem by taking previous runs into account, greatly speeding up the learning process and not experimenting with less than optimal hyperparameters~\cite{bayesian}.
Hyperband is a bandit-based approach to hyperparameter optimizations that can abandon or reduce the budget of an experimental run if it deems the results to be significantly poor~\cite{hyperband}.

\subsection{Bayesian Optimization}\label{subsection:background-bayesian}
The idea behind Bayesian optimization is that the hyperparameters of a network follow a probabilistic distribution $\text{P}(y | x)$ where $y$ is the score determined by some objective function and $x$ is a set of hyperparameters.
Based on this probabilistic model, an optimal set of hyperparameters can then be calculated.
Applying these hyperparameters to the actual model the objective function can then be used to calculate the actual scores.
The probabilistic distribution is the fine tuned and the process is repeated until the maximum number of iterations is reached.

Bayesian optimization relies on the following~\cite{bayesiantds}:
\begin{enumerate}
	\itemsep-1em
	\item A clearly defined hyperparameter domain of possible configurations $\chi$ over which to search,
	\item An objective function which calculates a score using a set of hyperparameters that we wish to minimize,
	\item A surrogate model that represents the probabilistic distribution of the hyperparameters,
	\item A selection function to evalute which hyperparameters should be evaluated next, and
	\item A history of score-hyperparameter pairs used to update the surrogate model .
\end{enumerate}
The domain must be chosen by the researcher and is usually defined with regards to previous knowledge or based on related works.
The objective function is calculated by running the actual model and calculating its error using a predetermined loss function.

The surrogate model used is essentially a mapping of hyperparameters to scores from the objective function.
An example for the surrogate model is the Tree-structured Parzen Estimator (TPE)~\cite{tpe}.
TPE takes advantage of Bayes' rule and represents the probability function $\text{P}(y~|~x)$ as
\begin{align}\label{eq:bohb-1}
	\text{P}(y~|~x) &= \frac{\text{P}(x~|~y) \cdot \text{P}(x)}{\text{P}(y)}\\
	\text{with } \text{P}(x~|~y) &=
	\begin{cases}
	l(x) &\text{when } y < y^* \\
	g(x) &\text{when } y \geq y^*
	\end{cases}
\end{align}
where $y^*$ is a threshold value. $l(x)$ is thus a probability density function formed from the set of observations $x^i \in \chi$ which have been performed such that the loss of the network running with hyperparameters $x^i$ is less than $y^*$. 
Conversely, $g(x)$ is the density function formed from the remaining observations.
$y^*$ is chosen by the algorithm such that $\text{P}(y<y^*) = \gamma$ with $\gamma$ being a value chosen by the researcher.

The selection function chooses which hyperparameters $x \in \chi$ should be chosen for each successive experiment and is commonly based on expected improvement~\cite{bayesiantds}
\begin{align}
	EI_{y^*}(x) &= \int_{-\infty}^{y^*} (y^* - y) p(y~|~x)dy
\end{align}
Here, $y^*$ is again a threshold value, $x$ is the proposed set of hyperparameters, $y$ is the actual value of the objective function using hyperparameters $x$ and $p(y~|~x)$ is the surrogate probability model expressing the probability of $y$ given $x$.
Substituting the values of $p(y~|~x)$ from the surrogate function we get
\begin{align}
	EI_{y^*}(x) &= \frac{\gamma y^* l(x) - l(x) \int_{-\infty}^{y^*}p(y)dy}{\gamma l(x) + (1 - \gamma)g(x)}\\
	&\propto \left(\gamma + \frac{g(x)}{l(x)} (1 - \gamma)\right)^{-1}
\end{align}
We can see from this function that the expected improvement is proportional to $\frac{l(x)}{g(x)}$ and thus, to maximize the expected improvement, samples should be drawn from the maximum value in $l(x)$.
The selected hyperparameters are then evaluated with regards to the objective function and the results used to update the surrogate model.

\subsection{Hyperband}\label{section:background-hyperband}
Hyperband is a learning strategy that "relies on a principled early-stopping strategy to allocate resources"~\cite{hyperband}. 
It is a variation of Successive Halving, and in fact uses successive halving in its inner loop.

Successive halving randomly samples $n$ hyperparameter sets in the search domain $\chi$.
It then evaluates all of these sets for $B$ iterations to calculate the validation loss.
The lowest performing half is then discarded and the remaining evaluated for a further $B$ iterations.
This is repeated until only 1 hyperparameter set remains.

Successive halving suffers from the $n \text{ vs } \frac{B}{n}$ problem, which is whether to train more configurations $n$ or to explore fewer, but with more resources $B$.
If a larger $n$ is chosen, configurations which are slower to converge might be killed off too quickly.
If a larger $\frac{B}{n}$ is chosen, then lower performing configurations may be given more resources, thus wasting resources that could otherwise be spent on higher performing configurations~\cite{hyperband}.

Hyperband attempts to solve this issue by considering several possible values of $n$ for a fixed $B$.
The pseudocode in Algorithm \ref{algorithm:hyperband} shows how Hyperband uses an outer loop, performing essentially a grid search with multiple values of $n$ and an inner loop utilizing a method similar to successive halving but with the top $\left\lfloor n_i/\eta\right\rfloor$ instead of the top half.

\begin{algorithm}
	\caption{\textsc{Hyperband} algorithm for hyperparameter optimization}\label{algorithm:hyperband}

	\SetKwInOut{Input}{Input}
	\SetKwInOut{Output}{Output}
	\SetKw{Initialize}{Initialize}
	
	\Input{$R,~\eta~ (\text{default} \eta = 3$)}
	\Output{Configuration with the smallest intermediate loss seen so far.}
	\Initialize{$s_{max} = \left\lfloor \log_\eta(R)\right\rfloor$, $B=(s_{max} + 1)R$}
	
	\For{$s \in s_{max}, s_{max} - 1,...,0$}{
		$n = \left\lceil \frac{B}{R} \frac{\eta^s}{(s+1)}\right\rceil$\\
		$r = R\eta^{-s}$\\
		$T = $ get\_hyperparameter\_configuration$(n)$\\
		\For{$i \in \{0,...,s\}$}{
			$n_i = \left\lfloor n \eta^{-i}\right\rfloor$ \\
			$r_i = r \eta^i$ \\
			$L = $\{run\_then\_return\_validation\_loss$(t,r_i) : t \in T$\}\\
			$T = $top\_k$(T,L, \left\lfloor n_i / \eta \right \rfloor)$\\
		}
	}
\end{algorithm}

\subsection{Bayesian Optimization with Hyperband}\label{section:background-bohb}
"BOHB: Robust and Efficient Hyperparameter Optimization at Scale" by Stefan Falkner, Aaron Klein, and Frank Hutter proposes a hyperparameter optimization approach which seeks to combine the benefits of both Bayesian optimization and Hyperband~\cite{bohb}.
This approach seeks to have strong anytime performance, strong final performance, effectively use parallel resources, be easily scalable, and be robust and flexible.

BOHB works by using Hyperband to determine the number of configurations to evaluate with a given budget, but replaces the random selection of configurations with a Bayesian model based search.
The process of using Bayesian optimization for configuration sampling can be seen in Algorithm \ref{algorithm:bohb}. In this algorithm, $l'(x)$ is the $l(x)$ component of the newly updated probability density function.

Even though Hyperband also provides strong anytime performance compared to random search and Bayesian optimization, Fakner et al. observed an improvement of over 55x against a random search at larger budgets, converging to the global optimum much faster than either Hyperband or Bayesian optimization alone~\cite{bohb}.

\begin{algorithm}
	\caption{\textsc{BOHB} algorithm for hyperparameter optimization}\label{algorithm:bohb}

	\SetKwInOut{Input}{Input}
	\SetKwInOut{Output}{Output}
	\SetKw{Initialize}{Initialize}
	
	\Input{observations $D$, fraction of random runs $\rho$, percentile $q$, number of samples $N_s$, minimum number of points $N_min$ to build a model, and bandwidth factor $b_w$}
	\Output{next configuration to evaluate}
	
	\If{rand() $< \rho$}{
		\Return{random configuration}
	}
	$b = \text{arg max }\{D_b:|D_b|\geq N_min + 2\}$\\
	\If{$b=\emptyset$}{
		\Return{random configuration	
	}}
	fit P$(x~|~y)$ according to \eqref{eq:bohb-1}\\
	draw $N_s$ samples according to $l'(x)$\\	
	\Return{sample with highest ratio $l(x) / g(x)$}
\end{algorithm}