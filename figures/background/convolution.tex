\definecolor{cred}{RGB}{155,47,47}
\definecolor{cblue}{RGB}{40,79,168}
\definecolor{cgreen}{RGB}{78,131,65}

% Elements of matrix 1
\def\elementsa{{2, 3, 3, 2, 2,
				7, 8, 5, 9, 0,
				2, 4, 7, 5, 4,
				4, 4, 0, 0, 6,
				9, 6, 5, 9, 3}}
% Elements of the convolution
\def\elementsb{{0, 1, 0, 
				1, 0, 1, 
				0, 1, 0}}

\def\elementsc{{19, 27, 12,
				21, 14, 20,
				14, 16, 20}}
\begin{figure}[t]
	\centering
	\resizebox{0.7\textwidth}{!}{
		\begin{tikzpicture}
			[x={(0.866cm,0.5cm)}, y={(-0.866cm,0.5cm)}, z={(0cm,1cm)}, scale=0.8]			
			% Matrix A
			\begin{scope}[canvas is xz plane at y=0,transform shape]
				
				\foreach \ii [count = \xi] in {1,2,3,4,5}{
		    		\foreach \jj  [count = \yi]in {1,2,3,4,5}{
		    			\pgfmathsetmacro{\nn}{int(\xi+5*\yi-5)}
						\node[cred,draw,minimum size=1cm] (n\nn-1) at (\ii,-\jj) {
							\pgfmathparse{\elementsa[(5*(\yi-1))+(\xi-1)]}\pgfmathresult
						};
					}
				}
				\node[cred,draw,minimum size=1cm,above=of n1-1.west,anchor=west] {matrix $A$};  % Matrix Label
			\end{scope}
			
			% Matrix B
			\begin{scope}[canvas is xz plane at y=-5.5,transform shape]
				\foreach \ii [count = \xi] in {1,2,3}{
		   			\foreach \jj  [count = \yi]in {1,2,3}{
				     	\pgfmathsetmacro{\nn}{int(\xi+3*\yi-3)}
				     	\pgfmathsetmacro{\val}{}]
						\node[cblue,draw,minimum size=1cm] (n\nn-2) at (\ii,-\jj) {
							\pgfmathparse{\elementsb[(3*(\yi-1))+(\xi-1)]}\pgfmathresult
						};
					}
				}
				\node[cblue,draw,minimum size=1cm,above=of n1-2.west,anchor=west] {matrix $B$};  % Matrix Label
			\end{scope} 
			
			% Matrix C
			\begin{scope}[canvas is xz plane at y=-9,transform shape]
				\foreach \ii [count = \xi] in {1,2,3}{
		   			\foreach \jj  [count = \yi]in {1,2,3}{
				    	\pgfmathsetmacro{\nn}{int(\xi+3*\yi-3)}
						\node[cgreen,draw,minimum size=1cm] (n\nn-3) at (\ii,-\jj) {
							\pgfmathparse{\elementsc[(3*(\yi-1))+(\xi-1)]}\pgfmathresult
						};
					}
				}
				\node[cgreen,draw,minimum size=1cm,above=of n1-3.west,anchor=west] {matrix $C$};  % Matrix Label
			\end{scope} 
			
			% Top of box
			\draw[fill=red!50,opacity=0.3] (n3-1.north east) -- (n3-2.north east) --(n1-3.north east)
			--(n1-3.north west)-- (n1-2.north west) --  (n1-1.north west)  ;
			
			% Bottom of box
			\draw[fill=red!50,opacity=0.3] (n13-1.south east) -- (n9-2.south east) --(n1-3.south east)
			--(n1-3.south west)-- (n7-2.south west)-- (n11-1.south west)  ;    

			% Right side of box			
			\draw[fill=red!50,opacity=0.3] (n3-1.north east) -- (n3-2.north east) --(n1-3.north east)
			--(n1-3.south east)-- (n9-2.south east)-- (n13-1.south east)  ;

			% Left side of box			
			\draw[fill=red!50,opacity=0.3] (n1-1.north west) -- (n1-2.north west) --(n1-3.north west)
			--(n1-3.south west)-- (n7-2.south west)-- (n11-1.south west)  ;      
		
		\end{tikzpicture}
	}
	\caption[Convolution operation]{An example of a how a convolution on matrix $A$ would result in matrix $C$. In this example, a convolution operation is being performed on $A$ with a kernel $B$ of size $3 \times 3$.}
\end{figure}